\documentclass[]{ctexart}

\usepackage{geometry}
\geometry{
	a4paper,
	total={170mm,257mm},
	left=20mm,
	top=20mm
}

\usepackage{float}
\usepackage{subfiles}
\usepackage[usenames,dvipsnames]{pstricks}
\usepackage{epsfig}
\usepackage{pst-grad} % For gradients
\usepackage{pst-plot} % For axes
\usepackage[space]{grffile} % For spaces in paths
\usepackage{etoolbox} % For spaces in paths
\makeatletter % For spaces in paths
\patchcmd\Gread@eps{\@inputcheck#1 }{\@inputcheck"#1"\relax}{}{}
\makeatother
%opening
\title{编译原理 \\ 作业 1}
\author{软件42 \\ 欧阳鹏程 \\ 2141601030 \\ 版权声明:BY-NC-SY}

\begin{document}

\maketitle

\begin{enumerate}
	\item[3.7] 构造下列正规式相应的DFA
	\begin{center}
		$1(0|1)^{*}101$ \\
		$1(1010^{*}|1(010)^{*}1)^{*}0$ \\
		$0^{*}10^{*}10^{*}10^{*}$ \\
		$(00|11)^{*}((01|10)(00|11)^{*}(01|10)(00|11)^{*})^{*}$
	\end{center}
	\subfile{3_7.tex}
	
	
	
	
	\item[3.9] 对下面情况给出DFA及正规表达式:
	\begin{enumerate}
		\item[(1)] \{0,1\}上的含有子串010的所有串;
		\item[(2)] \{0,1\}上不含字串010的所有串。
	\end{enumerate}
	\subfile{3_9.tex}
	
	
	
	
	\item[3.12] 将图3.18的(a)和(b)分别确定化和最少化。
	\begin{center}
	\psscalebox{0.7 0.7} % Change this value to rescale the drawing.
	{
		\begin{pspicture}(0,-2.6075)(14.38433,2.6075)
		\psline[linecolor=black, linewidth=0.04, doubleline=true, doublesep=0.12, dotsize=0.07055555cm 2.0,arrowsize=0.05291667cm 2.0,arrowlength=1.4,arrowinset=0.0]{cc->}(-0.11763889,0.3675)(1.4823611,0.3675)
		\pscircle[linecolor=black, linewidth=0.04, dimen=outer, doubleline=true, doublesep=0.02](1.882361,0.3675){0.4}
		\pscircle[linecolor=black, linewidth=0.04, dimen=outer](4.282361,0.3675){0.4}
		\psbezier[linecolor=black, linewidth=0.04, dotsize=0.07055555cm 2.0,arrowsize=0.05291667cm 2.0,arrowlength=1.4,arrowinset=0.0]{cc->}(1.4823611,0.3675)(1.4823611,1.1675)(1.4823611,1.1675)(1.882361,1.1675)(2.282361,1.1675)(2.282361,1.1675)(2.282361,0.3675)
		\psbezier[linecolor=black, linewidth=0.04, dotsize=0.07055555cm 2.0,arrowsize=0.05291667cm 2.0,arrowlength=1.4,arrowinset=0.0]{cc->}(2.282361,0.3675)(2.282361,1.1675)(2.6823611,1.1675)(3.0823612,1.1675)(3.482361,1.1675)(3.8823612,1.1675)(3.8823612,0.3675)
		\psbezier[linecolor=black, linewidth=0.04, dotsize=0.07055555cm 2.0,arrowsize=0.05291667cm 2.0,arrowlength=1.4,arrowinset=0.0]{cc->}(3.8823612,0.3675)(3.8823612,-0.0325)(3.8823612,-0.4325)(3.0823612,-0.4325)(2.282361,-0.4325)(2.282361,-0.0325)(2.282361,0.3675)
		\rput(1.882361,0.3675){0}
		\rput(4.282361,0.3675){1}
		\rput[b](1.882361,1.1675){a}
		\rput[b](3.0823612,1.1675){a,b}
		\rput[b](3.0823612,-0.4325){a}
		\psline[linecolor=black, linewidth=0.04, doubleline=true, doublesep=0.1, dotsize=0.07055555cm 2.0,arrowsize=0.05291667cm 2.0,arrowlength=1.4,arrowinset=0.0]{cc->}(6.282361,1.9675)(7.882361,1.9675)
		\pscircle[linecolor=black, linewidth=0.04, dimen=outer, doubleline=true, doublesep=0.02](8.282361,1.9675){0.4}
		\pscircle[linecolor=black, linewidth=0.04, dimen=outer](10.682361,1.9675){0.4}
		\pscircle[linecolor=black, linewidth=0.04, dimen=outer](13.082361,1.9675){0.4}
		\pscircle[linecolor=black, linewidth=0.04, dimen=outer](10.682361,-0.4325){0.4}
		\pscircle[linecolor=black, linewidth=0.04, dimen=outer](13.082361,-0.4325){0.4}
		\pscircle[linecolor=black, linewidth=0.04, dimen=outer, doubleline=true, doublesep=0.02](8.282361,-0.4325){0.4}
		\rput(8.282361,1.9675){0}
		\rput(8.282361,-0.4325){1}
		\rput(10.682361,1.9675){2}
		\rput(13.082361,1.9675){3}
		\rput(10.682361,-0.4325){4}
		\rput(13.082361,-0.4325){5}
		\psline[linecolor=black, linewidth=0.04, arrowsize=0.05291667cm 2.0,arrowlength=1.4,arrowinset=0.0]{->}(8.682361,1.9675)(10.282361,1.9675)
		\psline[linecolor=black, linewidth=0.04, arrowsize=0.05291667cm 2.0,arrowlength=1.4,arrowinset=0.0]{->}(8.282361,1.5675)(8.282361,-0.0325)
		\psline[linecolor=black, linewidth=0.04, arrowsize=0.05291667cm 2.0,arrowlength=1.4,arrowinset=0.0]{->}(8.682361,-0.4325)(10.282361,-0.4325)
		\psbezier[linecolor=black, linewidth=0.04, arrowsize=0.05291667cm 2.0,arrowlength=1.4,arrowinset=0.0]{->}(10.682361,1.5675)(10.682361,1.1675)(9.882361,0.3675)(9.882361,0.3675)(9.882361,0.3675)(9.482361,-0.0325)(8.682361,-0.4325)
		\psbezier[linecolor=black, linewidth=0.04, arrowsize=0.05291667cm 2.0,arrowlength=1.4,arrowinset=0.0]{->}(10.682361,-0.0325)(10.682361,0.3675)(9.882361,1.1675)(9.882361,1.1675)(9.882361,1.1675)(9.882361,1.1675)(8.682361,1.9675)
		\psbezier[linecolor=black, linewidth=0.04, arrowsize=0.05291667cm 2.0,arrowlength=1.4,arrowinset=0.0]{->}(8.282361,-0.8325)(7.8351474,-1.7269272)(7.4823613,-1.2325)(7.4823613,-1.2325)(7.4823613,-1.2325)(6.987934,-0.8797136)(7.882361,-0.4325)
		\psbezier[linecolor=black, linewidth=0.04, arrowsize=0.05291667cm 2.0,arrowlength=1.4,arrowinset=0.0]{->}(11.082361,1.9675)(11.482361,2.3675)(11.882361,2.3675)(11.882361,2.3675)(11.882361,2.3675)(12.282361,2.3675)(12.682361,1.9675)
		\psbezier[linecolor=black, linewidth=0.04, arrowsize=0.05291667cm 2.0,arrowlength=1.4,arrowinset=0.0]{->}(12.682361,1.9675)(12.282361,1.5675)(11.882361,1.5675)(11.882361,1.5675)(11.882361,1.5675)(11.482361,1.5675)(11.082361,1.9675)
		\psbezier[linecolor=black, linewidth=0.04, arrowsize=0.05291667cm 2.0,arrowlength=1.4,arrowinset=0.0]{->}(11.082361,-0.4325)(11.482361,-0.0325)(11.882361,-0.0325)(11.882361,-0.0325)(11.882361,-0.0325)(12.282361,-0.0325)(12.682361,-0.4325)
		\psbezier[linecolor=black, linewidth=0.04, arrowsize=0.05291667cm 2.0,arrowlength=1.4,arrowinset=0.0]{->}(12.682361,-0.4325)(12.282361,-0.8325)(11.882361,-0.8325)(11.882361,-0.8325)(11.882361,-0.8325)(11.482361,-0.8325)(11.082361,-0.4325)
		\psbezier[linecolor=black, linewidth=0.04, arrowsize=0.05291667cm 2.0,arrowlength=1.4,arrowinset=0.0]{->}(13.066143,-0.8262915)(13.049925,-0.8200831)(13.882361,-1.6325)(14.266143,-1.2262915264809635)(14.649924,-0.8200831)(13.506688,-0.43870848)(13.482361,-0.4325)
		\psbezier[linecolor=black, linewidth=0.04, arrowsize=0.05291667cm 2.0,arrowlength=1.4,arrowinset=0.0]{->}(13.082989,1.5280781)(13.6205635,0.6848619)(13.882361,1.1675)(13.920163,1.2132145417561795)(13.957965,1.2589291)(14.375331,1.6153818)(13.439342,1.96741)
		\rput[b](9.482361,1.9675){b}
		\rput[b](11.882361,2.3675){b}
		\rput[b](11.882361,1.5675){b}
		\rput[b](9.882361,1.1675){a}
		\rput[b](9.482361,-0.0325){a}
		\rput[b](11.882361,-0.0325){b}
		\rput[b](11.882361,-0.8325){b}
		\rput[b](13.882361,-1.2325){a}
		\rput[b](13.882361,1.1675){a}
		\rput[b](7.4823613,-1.2325){a}
		\rput[br](8.282361,0.7675){a}
		\rput[t](9.482361,-0.4325){b}
		\rput(11.082361,-2.4325){(b)}
		\rput(2.6823611,-2.4325){(a)}
		\end{pspicture}
	}
	\end{center}
	\subfile{3_12.tex}
	
	
	
	
	\item[3.14] 构造一个DFA,它接收$\Sigma=\{0,1\}$上所有满足如下条件的字符串:每个1都有0直接跟在右边。
	\subfile{3_14.tex}
		
\end{enumerate}

\end{document}
