\documentclass[]{ctexart}
\usepackage{subfiles}
\usepackage{float}
\usepackage{listings}
\usepackage{geometry}
\geometry{
	a4paper,
	total={170mm,257mm},
	left=20mm,
	top=20mm
}

\usepackage{afterpage}

%\newcommand \blankpage{
%	\null
%	\thispagestyle{empty}
%	\addtocounter{page}{-1}
%	\newpage
%}


%opening
\title{编译原理 \\ 作业 4}
\author{软件42 \\ 欧阳鹏程 \\ 2141601030 \\ 版权声明:Creative Commons BY-NC-SY}

\begin{document}

\maketitle

\begin{enumerate}
	\item[1.] 给出下面表达式的逆波兰表示(后缀式):
	\begin{enumerate}
		\item a * (-b + c)
		\item not A or not (C or not D)
		\item a + b * (c + d / e)
		\item (A and B) or (not C or D)
		\item -a + b * (-c + d)
		\item (A or B) and (C or not D and E)
		\item if (x + y) * z = 0 then (a + b) ↑ c else a ↑ b ↑ c
	\end{enumerate}
	\paragraph{答:}
	\begin{enumerate}
		\item a b - c + *
		\item A not C D not or not or
		\item a b c d e / + * +
		\item A B and C not D or or
		\item a - b c - d + * +
		\item A B or  C D not E and or and
		\item x y + z * 0 = a b + c $\uparrow$ a b c $\uparrow$ $uparrow$ ?
	\end{enumerate}
	
	\item[3.] 请将表达式 -(a + b) * (c + d) - (a + b + c) 分别表示成三元式、间接三元式和四元式序列。
	\paragraph{答:}
	\begin{table}[H]
		\centering
		\caption{三元式}
		\begin{tabular}{|c|c|c|c|}
			\hline
			~ & op & arg1 & arg2 \\\hline
			(0) & + & a & b \\
			(1) & + & c & d \\
			(2) & + & a & b \\
			(3) & + & c & (2) \\
			(4) & - & (0) & ~ \\
			(5) & * & (4) & (1) \\
			(6) & - & (5) & (3) \\\hline
		\end{tabular}
	\end{table}

	\begin{table}[H]
		\centering
		\caption{四元式}
		\begin{tabular}{|c|c|c|c|c|}
			\hline
			~ & op & arg1 & arg2 & result \\\hline
			(0) & + & a & b & $T_{1}$\\
			(1) & + & c & d & $T_{2}$\\
			(2) & + & a & b & $T_{3}$\\
			(3) & + & c & $T_{3}$ & $T_{4}$ \\
			(4) & - & $T_{1}$ & ~ & $T_{5}$\\
			(5) & * & $T_{5}$ & $T_{2}$ & $T_{6}$ \\
			(6) & - & $T_{6}$ & $T_{4}$ & $T_{7}$ \\\hline
		\end{tabular}
	\end{table}

	\begin{table}[H]
		\centering
		\caption{间接三元式}
		\begin{tabular}{|c|c|c|c|}
			\hline
			~ & op & arg1 & arg2 \\\hline
			(1) & + & a & b \\
			(2) & + & c & d \\
			(3) & + & c & (1) \\
			(4) & - & (1) & ~ \\
			(5) & * & (4) & (2) \\
			(6) & - & (5) & (3) \\\hline
		\end{tabular}
	\end{table}
	间接代码:(1)(2)(3)(1)(4)(5)(6)
		
	\item[5.] 按上课所讲方式写出四元式序列:
	\begin{center}
		A[i, j] := B[i, j] + C[A[k, l] + D[i + j]]
	\end{center}
	其中,数组 A, B 的规模均为 20 × 40,数组 C, D 的长度均为 80。
	\paragraph{答:}
	\begin{table}[H]
		\centering
		\caption{四元式}
		\begin{tabular}{|c|c|c|c|c|}
			\hline
			~ & op & arg1 & arg2 & result \\\hline
			(0) & - & B & 41 & $T_{1}$ \\
			(1) & * & i & 40 & $T_{2}$ \\
			(2) & + & $T_{2}$ & j & $T_{3}$ \\
			(3) & =[] & $T_{1}[T_{3}]$ & ~ & \textbf{$T_{4}$} \\
			(4) & - & A & 41 & $T_{5}$ \\
			(5) & * & k & 40 & $T_{6}$ \\
			(6) & + & $T_{6}$ & l & $T_{7}$ \\
			(7) & =[] & $T_{5}[T_{7}]$ & ~ & $T_{8}$ \\
			(8) & + & i & j & $T_{9}$ \\
			(9) & - & D & 1 & $T_{10}$ \\
			(10) & =[] & $T_{10}[T_{9}]$ & ~ & $T_{11}$ \\
			(11) & + & $T_{8}$ & $T_{11}$ & $T_{12}$ \\
			(12) & - & C & 1 & $T_{13}$ \\
			(13) & =[] & $T_{13}[T_{12}]$ & ~ & \textbf{$T_{14}$} \\
			(14) & - & A & 41 & $T_{15}$ \\
			(15) & * & i & 40 & $T_{16}$ \\
			(16) & + & $T_{16}$ & j & $T_{17}$ \\
			(17) & + & $T_{4}$ & $T_{14}$ & $T_{18}$ \\
			(18) & =[] & $T_{18}$ & ~ & $T_{15}[T_{17}]$ \\\hline
		\end{tabular}
	\end{table}
	
	\item[6.] 按上课所讲方式,写出布尔式 A or (B and not (C or D)) 的四元式序列。
	\paragraph{答:}
	\begin{itemize}
		\item 通常算法:
		\begin{table}[H]
			\centering
			\caption{四元式}
			\begin{tabular}{|c|c|c|c|c|}
				\hline
				~ & op & arg1 & arg2 & result \\\hline
				(0) & or & C & D & $T_{1}$ \\
				(1) & not & $T_{1}$ & ~ & $T_{2}$ \\
				(2) & and & B & $T_{2}$ & $T_{3}$ \\
				(3) & or & A & $T_{3}$ & $T_{4}$ \\\hline
			\end{tabular}
		\end{table}
	
		\item 采用优化措施:
		\begin{table}[H]
			\centering
			\caption{四元式}
			\begin{tabular}{|c|c|c|c|c|}
				\hline
				~ & op & arg1 & arg2 & result \\\hline
				(1) & jnz & A & ~ & 0 \\
				(2) & j & ~ & ~ & 3 \\
				(3) & jz & B & ~ & 1 \\
				(4) & j & ~ & ~ & 5 \\
				(5) & jnz & C & ~ & 3 \\
				(6) & j & ~ & ~ & 7 \\
				(7) & jnz & D & ~ & 5 \\
			\end{tabular}
		\end{table}
	\end{itemize}
	
	\item[7.] 按上课所讲方式,把下面的语句翻译成四元式序列:
	\begin{lstlisting}
	while A < C and B < D do
	if A = 1 then C := C + 1 else
	while A <= D do A := A + 2;
	\end{lstlisting}
	\paragraph{答:}
	\begin{table}[H]
		\centering
		\caption{四元式}
		\begin{tabular}{|c|c|c|c|c|}
			\hline
			~ & op & arg1 & arg2 & result \\\hline
			(1) & j>= & A & C & 0 \\
			(2) & j & ~ & ~ & 3 \\
			(3) & j>= & B & D & 1 \\
			(4) & j & ~ & ~ & 5 \\
			(5) & j!= & A & 1 & 9\\
			(6) & j & ~ & ~ & 7 \\
			(7) & + & C & 1 & $T_{1}$ \\
			(8) & = & $T_{1}$ & ~ & C \\
			(9) & j> & A & D & 1 \\
			(10) & j & ~ & ~ & 11 \\
			(11) & + & A & 2 & $T_{2}$ \\
			(12) & = & $T_{2}$ & ~ & A \\
			(13) & j & ~ & ~ & 9 \\
			(14) & j & ~ & ~ & 1 \\\hline
		\end{tabular}
	\end{table}
\end{enumerate}

\end{document}
