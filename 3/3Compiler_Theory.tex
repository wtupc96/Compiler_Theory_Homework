\documentclass[]{ctexart}
\usepackage{subfiles}
\usepackage{float}
\usepackage{amsmath}
\usepackage{amssymb}
\usepackage[normalem]{ulem}

\usepackage[usenames,dvipsnames]{pstricks}
\usepackage{epsfig}
\usepackage{pst-grad} % For gradients
\usepackage{pst-plot} % For axes
\usepackage[space]{grffile} % For spaces in paths
\usepackage{etoolbox} % For spaces in paths
\makeatletter % For spaces in paths
\patchcmd\Gread@eps{\@inputcheck#1 }{\@inputcheck"#1"\relax}{}{}
\makeatother

\usepackage{geometry}
\geometry{
	a4paper,
	total={170mm,257mm},
	left=20mm,
	top=20mm
}

\usepackage{afterpage}

%\newcommand \blankpage{
%	\null
%	\thispagestyle{empty}
%	\addtocounter{page}{-1}
%	\newpage
%}


%opening
\title{编译原理 \\ 作业 3}
\author{软件42 \\ 欧阳鹏程 \\ 2141601030 \\ 版权声明:Creative Commons BY-NC-SY}

\begin{document}

\maketitle

\begin{enumerate}
	\item[1] 令文法$G_{1}$为:
	\begin{align*}
		E &\to E+T | T \\
		T &\to T*F | F \\
		F &\to (E) | i
	\end{align*}
	证明E+T*F是它的一个句型,指出这个句型的所有短语,直接短语和句柄。
	%\subfile{1ans.tex}
	
	\item[2] 考虑下面的表格结构文法$G_{2}$:
	\begin{align*}
		S &\to a | \wedge | (T) \\
		T &\to T, S | S
	\end{align*}
	\begin{enumerate}
		\item \sout{给出(a, (a, a))和(((a, a), $\wedge$, (a)), a)的最左和最右推导。}
		\item \sout{指出(((a, a), $\wedge$, (a)), a)的规范归约及每一步的句柄。根据这个规范归约,给出“移进-归约”的过程,并给出它的语法树自下而上的构造过程。}
	\end{enumerate}
	
	\item[3]
	\begin{enumerate}
		\item 计算练习2文法$G_{2}$的FIRSTVT和LASTVT
		\item 计算$G_{2}$的优先关系。$G_{2}$是一个算符优先文法吗?
		\item 计算$G_{2}$的优先函数。
		\item \sout{给出输入串(a, (a, a))的算符优先分析过程。}
	\end{enumerate}
	%\subfile{3ans.tex}
	
	\item[5] 考虑文法
	\begin{align*}
		S &\to AS | b \\
		A &\to SA | a
	\end{align*}
	\begin{enumerate}
		\item 列出这个文法的所有LR(0)项目。
		\item 构造这个文法的LR(0)项目集规范族及识别活前缀的DFA。
		\item 这个文法是SLR的吗?若是,构造出它的SLR分析表。
		\item \sout{这个文法是LALR或LR(1)的吗?}
	\end{enumerate}
	%\subfile{5ans.tex}

	\item[7] 证明下面文法是SLR(1)但不是LR(0)的。
	\begin{align*}
		S &\to A \\
		A &\to Ab | bBa \\
		B &\to aAc | a | aAb
	\end{align*}
	\subfile{7ans.tex}
	
	\item[8] 证明下面的文法
	\begin{align*}
		S &\to AaAb | BbBa \\
		A &\to \epsilon \\
		B &\to \epsilon
	\end{align*}
	\sout{是LL(1)的但}不是SLR(1)的。
	%\subfile{8ans.tex}
	
	\item[9] 证明下面文法:
	\begin{align*}
		S &\to Aa | bAc | Bc | bBa \\
		A &\to d
	\end{align*}
	\sout{是LALR(1)的但}不是SLR(1)的。
\end{enumerate}

\end{document}
