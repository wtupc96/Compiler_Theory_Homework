\documentclass[]{ctexart}
\usepackage{subfiles}
\usepackage{amsmath}
\usepackage[normalem]{ulem}

\usepackage{geometry}
\geometry{
	a4paper,
	total={170mm,257mm},
	left=20mm,
	top=20mm
}

\usepackage{afterpage}

%\newcommand \blankpage{
%	\null
%	\thispagestyle{empty}
%	\addtocounter{page}{-1}
%	\newpage
%}


%opening
\title{编译原理 \\ 作业 2}
\author{软件42 \\ 欧阳鹏程 \\ 2141601030 \\ 版权声明:BY-NC-SY}

\begin{document}

\maketitle

\begin{enumerate}
	\item 考虑下述文法 $G_1$:
	
	\begin{align*}
	S & \to a | \wedge | (T) \\
	T & \to T,S | S
	\end{align*}
	
	\begin{enumerate}
		\item 消去 $G_1$ 的左递归。\sout{然后,对每个非终结符,写出不带回溯的递归子程序。}
		\item 经改写后的文法是否是 LL(1) 的?给出它的预测分析表。
		
	\end{enumerate}
	%\blankpage
	\newpage
	
	\item 对下列的文法 G:
	
	\begin{align*}
	E & \to TE' \\
	E' & \to +E | \epsilon \\
	T & \to FT' \\
	T' & \to T | \epsilon \\
	F & \to PF' \\
	F' & \to *F' | \epsilon \\
	P & \to (E) | a | b | \wedge
	\end{align*}
	
	\begin{enumerate}
		\item 计算这个文法的每个非终结符的 FIRST 和 FOLLOW。
		\item 证明这个文法是 LL(1) 的。
		\item 构造它的预测分析表。
		\item \sout{构造它的递归下降分析程序。}
	\end{enumerate}
	%\blankpage
	%\blankpage
	\newpage
	
	\null
	\newpage
	
	\item 下面文法中,哪些是 LL(1) 的,说明理由。
	
	\begin{enumerate}
		\item \begin{align*}
		S & \to ABc \\
		A & \to a | \epsilon \\
		B & \to b | \epsilon
		\end{align*}
		
		\item \begin{align*}
		S & \to Ab \\
		A & \to a | B | \epsilon \\
		B & \to b | \epsilon
		\end{align*}
		
		\item \begin{align*}
		S & \to ABBA \\
		A & \to a | \epsilon \\
		B & \to b | \epsilon
		\end{align*}
		
		\item \begin{align*}
		S & \to aSe | B \\
		B & \to bBe | C \\
		C & \to cCe | d
		\end{align*}
		
	\end{enumerate}
	%\blankpage
	%\blankpage
	\newpage
	
	\null
	\newpage
		
	\item 对下面文法:
	
	\begin{align*}
	Expr & \to -Expr \\
	Expr & \to (Expr) | Var\ ExprTail \\
	ExprTail & \to -Expr | \epsilon \\
	Var & \to id\ VarTail \\
	VarTail & \to (Expr) | \epsilon
	\end{align*}
	
	\begin{enumerate}
		\item 构造 LL(1) 分析表。
		\item \sout{给出对句子 id - - id((id)) 的分析过程。}
	\end{enumerate}
	
\end{enumerate}

\end{document}
